Dies ist ein Beispiel der Anwendung des Gaußalgorithmus.
Wir nehmen als  Ausgangsmatrix die $m\times n$ Matrix $A$ welch zur $m \times n$ Matrix $\tilde{A}$  
\begin{eqnarray*}
A&=&\begin{pmatrix}
0 & 0 & 1 & 2 & 9 \\ 
0 & 3 & 4 & 5 & 9 \\ 
0 & 6 & 7 & 8 & 9 \\ 
0 & 9 & 9 & 9 & 9
\end{pmatrix} 
\leadsto
\begin{pmatrix}
0 & 3 & 4 & 5 & 9 \\ 
0 & 0 & 1 & 2 & 9 \\ 
0 & 6 & 7 & 8 & 9 \\ 
0 & 9 & 9 & 9 & 9
\end{pmatrix}
\leadsto 
\begin{pmatrix}
 0 & 3 & 4 & 5 & 9 \\ 
 0 & 0 & 1 & 2 & 9 \\ 
 0 & 0 & -1 & -2 & -9 \\ 
 0 & 0 & -3 & -6 & -18
\end{pmatrix}\\
&\leadsto &
\begin{pmatrix}
  0 & 3 & 4 & 5 & 9 \\ 
  0 & 0 & 1 & 2 & 9 \\ 
  0 & 0 & 0 & 0 & 0 \\ 
  0 & 0 & 0 & 0 & 9
  \end{pmatrix}
\leadsto
\begin{pmatrix}
  0 & 3 & 4 & 5 & 9 \\ 
  0 & 0 & 1 & 2 & 9 \\ 
  0 & 0 & 0 & 0 & 9 \\ 
  0 & 0 & 0 & 0 & 0
 \end{pmatrix}  
 \end{eqnarray*}
 
 Im ersten Schritt wurden die erste und zweite Zeile vertauscht um den ersten Pivot zu erhalten.
 Danach wurden zu der dritten und vierten Zeile Vielfache der ersten addiert, was die Einträge $a_{3,2}$ und $a_{4,2}$ eliminierte. Der Gauß-Algorithmus wurde nun erneut an der Matrix $A_2=A_{2-3,3-5}$ ausgeführt wodurch man die vierte Matrix erhält, in welcher nur noch die dritte und vierte Zeile Vertauscht werden müssen um Matrix $\tilde{A}$ in ZSF zu erhalten.
