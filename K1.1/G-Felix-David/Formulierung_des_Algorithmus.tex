\subsubsection{Der Gauß-Algorithmus allgemein formuliert}
\begin{Algo}[Gauß-Algorithmus]
	Hier wird der Gauß-Algorithmus in seiner allgemeinen Form definiert und er kann so zb. im Computer umgesetzt werden.
	\begin{description}
		\item[Eingabe:] $m \times n $-Matrix $A$ , $a_j$ sei die $j$-te Spalte von $A$
		\item[Ausgabe:] $\tilde{A}$ in Zeilenstufenform (ZSF)
	\end{description}
	
	\begin{enumerate}
		\item Zuerst wird geprüft ob die Matrix leer ist also ob $A=0$. Ist dies der Fall so ist $A$ bereits in ZSF.
		\item Nun wird die Zählvariable $j$ auf $j=1$ gesetzt.
		\item Wenn die $j$-te Spalte leer ist also $a_j=0$ dann wird $j$ um eins erhöht $j+1\rightarrow j$ und dieser Schritt wiederholt.
		\item $k$ wird definiert als der Index des ersten Eintrags in der $j$-ten Spalte welcher ungleich Null d.h. $k:=\min\{i|a_{ij}\neq0\}$.
		\item Ist nun $k\neq1$ so muss die $k$-te Zeile mit der ersten vertauscht werden um im nächsten Schritt fortzufahren zu können.
		\item Da $k\neq0$ gilt ist es nun möglich dank der Zeilenaddition und dem Faktor $\lambda=\frac{-a_{ij}}{\bar{a}_{1j}}$ alle Einträge unter $a_1j$ in den Zeilen $2\leq{i}\leq{m}$ durch Multiplikation mit $Q_i^1(\lambda)$ zu eliminieren.
		\item Hiernach werden die Schritte 1 bis 6 mit der Untermatrix $A_2:=(a_{pq})_{\substack{2\leq p\leq m\\ j+1 \leq q \leq n}}$ wiederholt solange $m>1$
		\item $A$ ist in ZSF. 
	\end{enumerate}
\end{Algo}
\subsubsection{Terminierung des Gauß-Algorithmus}
	Der Gauß-Algorithmus terminiert auf jeden Fall da mit jedem Durchlauf $m$ und $n$ abnehmen und streng monoton fallend sind und da die $m \times n$ Matrix $A$ endlich groß ist. 
