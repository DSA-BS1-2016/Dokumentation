Der Gauß-Algorithmus kann verwendet werden, um die Inverse einer quadratischen Matrix $M$ zu berechnen.
Dies ist folgendermaßen möglich:

\begin{enumerate}
\item Man definiere zuerst eine neue Matrix $A$ folgendermaßen:
$A = (M|E_m)$
Dies bedeutet, dass an die Matrix $M$ eine Einheitsmatrix $E_m$ der Größe von $M$ von rechts angehängt wird. 
\item Diese neu definierte Matrix wird nun mithilfe des Gauß-Algorithmus so umgeformt, dass am Anfang der Matrix eine Einheitsmatrix $E_m$ steht.
\item Ist dies nicht möglich, so kann die Matrix $M$ nicht invertiert werden.
\item Die daraus folgende Matrix $A$ ist nun folgendermaßen aufgebaut: \[ A = (E_m|M^{-1}) \]
\end{enumerate}

\begin{proof}
Um eine Matrix $M$ in die Form $(E_m|B)$ zu bringen, wendet der Gauß-Algorithmus ein Produkt von Elementarmatrizen $(E_k \cdots E_1)$ an.

Zudem liefert der Gauß-Algorithmus bei der Umformung von $(M|E_m)$ die Matrix $(E_m|B)$, d.h.: 
\begin{align*}
  (E_m|B) &= (E_k \cdots E_1) \cdot (M|E_m)\\ 
  (E_m|B) &= ((E_k \cdots E_1) \cdot M | (E_k \cdots E_1))
\end{align*}

Es lässt sich nun wie folgt ab lesen:
\begin{align*}
 (E_k \cdots E_1) &= B \\
(E_k \cdots E_1) \cdot M &= E_m
\end{align*}

Da nun laut obiger Formel $(E_k \cdots E_1) = B$ gilt, lässt sich nun in der zweiten Formel $(E_k \cdots E_1)$ durch $B$ ersetzten. Hieraus erhält man:
\[ B\cdot M = E_m \]

Da dies jedoch äquivalent ist zur Definition der Inverse einer Matrix, gilt:
\[ B=M^{-1}\]

\textbf{TODODODO}

\end{proof}