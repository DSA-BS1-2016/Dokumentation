Nebst seinen vielen Anwendungen zur Restrukturierung von Matrizen in eine Zeilen-Stufen-Form kann der Gauss-Algorithmus auch verwendet werden, um die Inverse einer quadratischen Matrix $M$ zu berechnen.
Dies ist folgendermaßen möglich:

\begin{itemize}
\item Man definiere zuerst eine neue Matrix $A$ folgendermaßen:
$A = (M|E_m)$
\item Diese neu definierte Matrix wird nun mithilfe des Gauss-Algorithmus so umgeformt, dass am Anfang der Matrix eine Einheitsmatrix $E_m$ steht.
\begin{itemize}
\item Ist dies nicht möglich, so kann die Matrix $M$ nicht invertiert werden.
\end{itemize}
\item Die neue Matrix $A_{\tiny Ergebnis}$ ist nun folgendermaßen aufgebaut: $A_{\tiny Ergebnis} = (E_m|M^{-1})$
\end{itemize}
Entfernt man nun die ersten $m$ Spalten der Matrix $A_{\tiny Ergebnis}$, so ist die verbleibende Matrix $A_{\tiny Schnitt}$ gleich der Inversen der Matrix $M$, d.h. $M^{-1}$

\begin{Bew}

\end{Bew}