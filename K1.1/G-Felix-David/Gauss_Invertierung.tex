Nebst seinen vielen Anwendungen zur Restrukturierung von Matrizen in eine Zeilen-Stufen-Form kann der Gauss-Algorithmus auch verwendet werden, um die Inverse einer quadratischen Matrix $M$ zu berechnen.
Dies ist folgendermaßen möglich:

\begin{itemize}
\item Man definiere zuerst eine neue Matrix $A$ folgendermaßen:
$A = (M|E_m)$
\item Diese neu definierte Matrix wird nun mithilfe des Gauss-Algorithmus so umgeformt, dass am Anfang der Matrix eine Einheitsmatrix $E_m$ steht.
\begin{itemize}
\item Ist dies nicht möglich, so kann die Matrix $M$ nicht invertiert werden.
\end{itemize}
\item Die daraus folgende Matrix $A_{Ergebnis}$ ist nun folgendermaßen aufgebaut: $A_{Ergebnis} = (E_m|M^{-1})$
\end{itemize}
Entfernt man nun die ersten $m$ Spalten der Matrix $A_{Ergebnis}$, so ist die verbleibende Matrix gleich der Inversen der Matrix $M$, d.h. $M^{-1}$

\begin{proof}
Um eine Matrix $M$ in die ZSF zu brigen, wendet der Gauss-Algorithmus eine Verkettung von Elementarmatrizen $(E_1 \cdots E_k)$ an.

Zudem liefert der Gauss-Algorithmus bei der Umformung von $(M|E_m)$ die Matrix $(E_m|B)$, d.h.: 
\begin{eqnarray*}
 (E_1 \cdots E_k) \cdot (M|E_m) &=& (E_m|B) \\
 \Leftrightarrow 
 ((E_1 \cdots E_k) \cdot A | (E_1 \cdots E_k)) &=& (E_m|B)
\end{eqnarray*}

Diese Formel lässt sich nun wie folgt aufspalten:
\begin{eqnarray*}
 (E_1 \cdots E_k) &=& B \\
(E_1 \cdots E_k) \cdot A &=& E_m
\end{eqnarray*}

Durch Ersetzung von $(E_1\cdots E_k)$ mit $B$, da diese Ausdrücke äquivalent sind (siehe oben), erhält man:
\[ B\cdot A = E_m \]

Da dies jedoch äquivalent ist zur Definition der Inverse einer Matrix gilt:
\[ B=A^{-1}\]

\end{proof}