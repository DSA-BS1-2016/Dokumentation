
Für den Gauss-Algorithmus sind einige basische Zeilenumformungen innerhalb einer Matrix notwendig. Diese sind:

\begin{itemize}
\item Die Zeilenaddition
\item Die Zeilenmultiplikation
\item Die Zeilenvertauschung
\end{itemize}

Um dieser verschiedenen Operationen auszuführen lässt sich die Matrix-Matrix-Multiplikation anwenden. Hierfür lassen sich diverse Matrizen definieren.
Um jedoch eben jene Matrizen zu definieren, muss zuerst die Matrix $E^i_j$ definiert werden als:

\subsubsection{Elementarmatrix $E^i_j$}
\begin{Def} Elementarmatrix $E^i_j$
\begin{eqnarray}
	E^i_j &=& m \times m \\
 	E^i_j &=& 
	 	\begin{pmatrix}
	 	0 & \cdots & 0 \\
	 	\vdots & e^i_j = 1 & \vdots \\
	 	0 & \cdots & 0 
 	\end{pmatrix}
\end{eqnarray}
\end{Def}

% Raggedright um linksbündig zu bleiben
\raggedright Nun lassen sich die verschiedenen Zeilenumformungen folgendermaßen definieren:

\subsubsection{Zeilenaddition $Q^i_j(\lambda)$}
Bei der Zeilenaddition wird zur Zeile i die Zeile j mit Vorfaktor $\lambda$ hinzugefügt.
Die dazugehörige Matrix $Q^i_j(\lambda)$ ist folgendermaßen definiert:
\begin{Def} Zeilenaddition:
\begin{eqnarray}
	M_T &=& M \cdot Q^i_j(\lambda) \\
	Q^i_j(\lambda) & = & E_m + \lambda \cdot E^i_j \\
	Q^i_j(\lambda) & = & 
	\begin{pmatrix}
	1 & \cdots & 0 \\ 
	\vdots &  q^i_j = \lambda & \vdots \\  
	0 & \cdots & 1
	\end{pmatrix} 
\end{eqnarray}
\end{Def}

\subsubsection{Zeilenmultiplikation $S^i(\lambda)$}
Bei der Zeilenmultiplikation wird die Zeile i in M mit dem Faktor $\lambda$ multipliziert. Andere Zeilen werden hierbei nicht verändert. Die dazugehörige Matrix $S^i(\lambda)$ ist folgendermaßen definiert:
\begin{Def}
\begin{eqnarray}
	M_T &=& M \cdot S^i(\lambda) \\
	S^i(\lambda) &=& E_m + (\lambda - 1) \cdot E^i_i \\
	S^i(\lambda) &=& 
	\begin{pmatrix}
	1 & \cdots & 0 \\ 
	\vdots & \lambda & \vdots \\ 
	0 & \cdots & 1
	\end{pmatrix} 
\end{eqnarray}
\end{Def}

\subsubsection{Zeilenvertauschung $P^i_j$}
Bei der Zeilenvertauschung werden die Zeilen i und j miteinander vertauscht. Die daraus entstehende Matrix enthält noch beide Zeilen wie zuvor, jedoch an anderen Positionen. Die dazugehörige Matrix $P^i_j$ ist folgendermaßen definiert:
\begin{Def} Zeilenvertauschung:
\begin{eqnarray}
	M_T &=& M \cdot P_j^i(\lambda) \\
	P_j^i(\lambda) &=& E_m - E^i_i - E^j_j + E^j_i + E^i_j \\
	P^i_j &=& 
	\begin{pmatrix}
	1 & \cdots & \cdots & 0 \\ 
	\vdots & E^i_i = 0 & E^i_j = 1 &  \\ 
	\vdots & E^j_i = 1 & E^j_j = 0 &  \\ 
	0 & \cdots & \cdots & 1
	\end{pmatrix} 
\end{eqnarray}
\end{Def}