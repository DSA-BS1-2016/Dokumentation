
Für den Gauss-Algorithmus sind einige elementare Zeilenumformungen innerhalb einer Matrix notwendig.
Um diese verschiedenen Operationen auszuführen lässt sich die Matrix-Matrix-Multiplikation anwenden. Hierfür lassen sich drei verschiedene, im folgenden aufgeführte Matrizen definieren.
Um jedoch eben jene Matrizen zu definieren, muss zuerst die Matrix $E^j_i$ definiert werden:

\subsubsection{Matrix $E^j_i$}
\begin{Def} Matrix $E^j_i$
\begin{align*}
 	E^j_i &=
	 	\begin{pmatrix}
	 	0 & \cdots & 0 \\
	 	\vdots & e_{ij} = 1 & \vdots \\
	 	0 & \cdots & 0 
 	\end{pmatrix}
\end{align*}
\end{Def}

% Raggedright um linksbündig zu bleiben
\vspace{8pt}

\raggedright Nun lassen sich die verschiedenen Zeilenumformungen folgendermaßen definieren:
\subsubsection{Zeilenaddition $Q^j_i(\lambda)$}
\begin{Def} Zeilenaddition:
\begin{align*}
	Q^j_i(\lambda) & = E_m + \lambda \cdot E^j_i \\
	Q^j_i(\lambda) & = 
	\begin{pmatrix}
	1 & \cdots & 0 \\ 
	\vdots &  q_{ij} = \lambda & \vdots \\  
	0 & \cdots & 1
	\end{pmatrix} 
\end{align*}
\end{Def}
\raggedright
Wird diese Matrix an eine Matrix $M$ von links heran multipliziert, so ist in der Matrix des Ergebnisses zur Zeile $i$ die Zeile $j$ mit Vorfaktor $\lambda$ hinzugefügt worden.

\subsubsection{Zeilenmultiplikation $S^i(\lambda)$}
\begin{Def}
\begin{align*}
	S^i(\lambda) &=& E_m + (\lambda - 1) \cdot E^i_i \\
	S^i(\lambda) &=& 
	\begin{pmatrix}
	1 & \cdots & 0 \\ 
	\vdots & s_{ii} = \lambda & \vdots \\ 
	0 & \cdots & 1
	\end{pmatrix} 
\end{align*}
\end{Def}
\raggedright 
Wird diese Matrix an eine Matrix $M$ von links heran multipliziert, so wird die Zeile $i$ in $M$ mit dem Faktor $\lambda$ multipliziert. Andere Zeilen werden hierbei nicht verändert.

\subsubsection{Zeilenvertauschung $P^j_i$}
\begin{Def} Zeilenvertauschung:
\begin{align*}
	P^j_i &= E_m - E^i_i - E^j_j + E^j_i + E^j_i \\
	P^j_i &= 
	\begin{pmatrix}
	1 & \cdots & \cdots & 0 \\ 
	\vdots & E_{ii} = 0 & E_{ij} = 1 &  \\ 
	\vdots & E_{ji} = 1 & E_{jj} = 0 &  \\ 
	0 & \cdots & \cdots & 1
	\end{pmatrix} 
\end{align*}
\end{Def}
\raggedright 
Wird diese Matrix an eine Matrix $M$ von links heran multipliziert, so werden die Zeilen $i$ und $j$ miteinander vertauscht. Die daraus entstehende Matrix enthält noch beide Zeilen wie zuvor, jedoch an anderen Positionen.