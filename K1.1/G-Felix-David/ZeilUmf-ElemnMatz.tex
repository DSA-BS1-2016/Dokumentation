
Für den Gauss-Algorithmus sind einige basische Zeilenumformungen innerhalb einer Matrix notwendig. Diese sind:

\begin{itemize}
\item Die Zeilenvertauschung
\item Die Zeilenaddition
\item Die Zeilenmultiplikation
\end{itemize}

Um dieser verschiedenen Operationen auszuführen lässt sich die Matrix-Matrix-Multiplikation anwenden. Hierfür lassen sich diverse Matrizen definieren.
Um jedoch eben jene Matrizen zu definieren, muss zuerst die Matrix $E^i_j$ definiert werden als:
\begin{Def}
\[ E &=& \]
\end{Def}

\begin{Def} Zeilenvertauschung:

\begin{eqnarray}
	M_T &=& M \cdot P_j^i(\lambda) \\
	P_j^i(\lambda) &=& E
\end{eqnarray}

\end{Def}