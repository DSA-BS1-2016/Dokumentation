
Für den Gauss-Algorithmus sind einige basische Zeilenumformungen innerhalb einer Matrix notwendig. Diese sind:

\begin{itemize}
\item Zeilenaddition
\item Zeilenmultiplikation
\item Zeilenvertauschung
\end{itemize}

Um diese verschiedenen Operationen auszuführen lässt sich die Matrix-Matrix-Multiplikation anwenden. Hierfür lassen sich drei verschiedene, im folgenden aufgeführte Matrizen $Q^i_j(\lambda)$, $S^i(\lambda)$ und $P^i_j$ definieren.
Um jedoch eben jene Matrizen zu definieren, muss zuerst die Matrix $E^i_j$ definiert werden als:


\textbf{TODODODO}
\subsubsection{Matrix $E^i_j$}
\begin{Def} Matrix $E^i_j$
\begin{align*}
 	E^i_j &=
	 	\begin{pmatrix}
	 	0 & \cdots & 0 \\
	 	\vdots & e_{ij} = 1 & \vdots \\
	 	0 & \cdots & 0 
 	\end{pmatrix}
\end{align*}
\end{Def}

% Raggedright um linksbündig zu bleiben
\vspace{8pt}

\raggedright Nun lassen sich die verschiedenen Zeilenumformungen folgendermaßen definieren:
\subsubsection{Zeilenaddition $Q^i_j(\lambda)$}
Bei der Zeilenaddition wird zur Zeile $i$ die Zeile $j$ mit Vorfaktor $\lambda$ hinzugefügt.
Die Matrix $Q^i_j(\lambda)$ ist folgendermaßen definiert:
\begin{Def} Zeilenaddition:
\begin{align*}
	Q^i_j(\lambda) & = & E_m + \lambda \cdot E^i_j \\
	Q^i_j(\lambda) & = & 
	\begin{pmatrix}
	1 & \cdots & 0 \\ 
	\vdots &  q_{ij} = \lambda & \vdots \\  
	0 & \cdots & 1
	\end{pmatrix} 
\end{align*}
\end{Def}
\raggedright
Um nun diese Elementarmatrix auf eine Matrix M anzuwenden, muss sie von links mit $M$ multipliziert werden: $Q^i_j(\lambda) \cdot M$
Das Ergebnis dieser Multiplikation ist nun eine neue Matrix, welche den Dimensionen der Matrix $M$ entspricht und dieselben Zeilen hat wie $M$, jedoch mit dem Unterschied dass zur Zeile $i$ die Zeile $j$ mit Vorfaktor $\lambda$ hinzugefügt wurde.

\subsubsection{Zeilenmultiplikation $S^i(\lambda)$}
Bei der Zeilenmultiplikation wird die Zeile $i$ in $M$ mit dem Faktor $\lambda$ multipliziert. Andere Zeilen werden hierbei nicht verändert. Die Matrix $S^i(\lambda)$ ist folgendermaßen definiert:
\begin{Def}
\begin{align*}
	S^i(\lambda) &=& E_m + (\lambda - 1) \cdot E^i_i \\
	S^i(\lambda) &=& 
	\begin{pmatrix}
	1 & \cdots & 0 \\ 
	\vdots & s_{ii} = \lambda & \vdots \\ 
	0 & \cdots & 1
	\end{pmatrix} 
\end{align*}
\end{Def}
\raggedright Um nun diese Elementarmatrix auf eine Matrix $M$ anzuwenden muss sie, wie bekannt, von links mit $M$ multipliziert werden: $S^i(\lambda) \cdot M$. 
Das Ergebnis dieser Multiplikation ist eine Matrix mit den Dimensionen von $M$, sowie den selben Zeilen, mit jedoch dem Unterschied dass die Zeile $i$ mit einem Vorfaktor $\lambda$ multipliziert wurde.

\subsubsection{Zeilenvertauschung $P^i_j$}
Bei der Zeilenvertauschung werden die Zeilen $i$ und $j$ miteinander vertauscht. Die daraus entstehende Matrix enthält noch beide Zeilen wie zuvor, jedoch an anderen Positionen. Die dazugehörige Matrix $P^i_j$ ist folgendermaßen definiert:
\begin{Def} Zeilenvertauschung:
\begin{align*}
	P_j^i(\lambda) &=& E_m - E^i_i - E^j_j + E^j_i + E^i_j \\
	P^i_j &=& 
	\begin{pmatrix}
	1 & \cdots & \cdots & 0 \\ 
	\vdots & E_{ii} = 0 & E_{ij} = 1 &  \\ 
	\vdots & E_{ji} = 1 & E_{jj} = 0 &  \\ 
	0 & \cdots & \cdots & 1
	\end{pmatrix} 
\end{align*}
\end{Def}
\raggedright Um nun Zeile $i$ und $j$ in einer Matrix $M$ zu vertauschen muss die Elementarmatrix $P^i_j$ von links an $M$ multipliziert werden: $P^i_j \cdot M$
Das Ergebnis dieser Multiplikation ist eine Matrix mit den Dimensionen von $M$ sowie den gleichen Zeilen, mit dem Unterschied dass die Zeilen $i$ und $j$ miteinander vertauscht wurden.