
Für den Gauss-Algorithmus sind einige basische Zeilenumformungen innerhalb einer Matrix notwendig. Diese sind:

\begin{itemize}
\item Die Zeilenvertauschung
\item Die Zeilenaddition
\item Die Zeilenmultiplikation
\end{itemize}

Um dieser verschiedenen Operationen auszuführen lässt sich die Matrix-Matrix-Multiplikation anwenden. Hierfür lassen sich diverse Matrizen definieren.
Um jedoch eben jene Matrizen zu definieren, muss zuerst die Matrix $E^i_j$ definiert werden als:

\begin{Def} Elementarmatrix $E^i_j$
\begin{eqnarray}
	E^i_j &=& m \times m \\
 	E^i_j &=& 
	 	\begin{bmatrix}
	 	0 & \cdots & 0 \\
	 	\vdots & e^i_j = 1 & \vdots \\
	 	0 & \cdots & 0 
 	\end{bmatrix}
\end{eqnarray}
\end{Def}

Nun lassen sich die verschiedenen Zeilenumformungen folgendermaßen definieren:

\begin{Def} Zeilenvertauschung:
\begin{eqnarray}
	M_T &=& M \cdot P_j^i(\lambda) \\
	P_j^i(\lambda) &=& E_m - E^i_i - E^j_j + E^j_i + E^i_j \\
	P^i_j &=& \begin{bmatrix}
	1 & \cdots & \cdots & 0 \\ 
	\vdots & E^i_i = 0 & E^i_j = 1 &  \\ 
	\vdots & E^j_i = 0 & E^j_j = 0 &  \\ 
	0 & \cdots & \cdots & 1
	\end{bmatrix} 
\end{eqnarray}
\end{Def}

\begin{Def} Zeilenaddition:
\begin{eqnarray}
M_T &=& M \cdot Q^i_j(\lambda) 
\end{eqnarray}
\end{Def}