Das Problem der Eisproduktionsplanung befasst sich mit folgendem Szenario:

Man besitzt eine Eis-Firma, welche über den gesamten Verlauf des Jahres Eis produzieren soll. Da man jedoch keine Über- oder Unterproduktion, und damit Verluste, in Kauf nehmen möchte, versucht man mithilfe von Vorhersagen für das folgende Jahr seine Eisproduktion hieran anzupassen. Eine mögliche Vorhersage sieht folgendermaßen aus:

\centering
\includegraphics[width=\textwidth]{Grafiken/Eiscreme.png}

\raggedright
Bei der Erstellung des Linearen Problems muss nun folgendes beachtet werden:

\begin{itemize}
\item Die Änderung der Produktionsmenge von Monat A zu Monat B kostet 50€ pro Tonne.
\item Eis lässt sich lagern, jedoch kostet dies 20€ pro Tonne pro Monat.
\item Die Produktion von einer Tonne Eis kostet 1€
\item Es darf keine Unterproduktion vorliegen, d.h. der Bedarf muss gedeckt werden.
\item Zu Anfang des Jahres gibt es kein Eis, und zu Ende des Jahres darf ebenso kein Eis mehr vorhanden sein.
\end{itemize}

Dies liefert folgende Bedingungen:
\begin{align*}
 \min: &\sum (1 \cdot p_m + 50 \cdot (ci_m + cd_m) + 20 \cdot l_m) \\
 p_m + l_m &\geq n_m  \\
 l_m &= p_{m-1} + l_{m-1} - n_{m-1} \\
 \sum p_m &= \sum n_m \\
 c_m &= p_{m} - p_{m-1} \\
 c_m &= ci_m - cd_m \\
 ci_m &\geq 0		\\
 cd_m &\geq 0
\end{align*}

Im folgenden werden nun diese (Un)Gleichungen genauer erläutert:

\begin{itemize}
\item Für einen Monat $m$ in $M$, wobei $M = \{1, 2, \cdots, 12\}$ ist, muss die Produktionsmenge $p_m$, sowie die Menge der gelagerten Eiscreme $l_m$ größer als die Nachfrage $n_m$ sein:
\[ p_m + l_m \geq n_m \]
\item Die Menge des in einem Monat gelagerten Eis ist gleich der in vorherigen Monat übrig gebliebenen Menge Eis:
\[ l_m = p_{m-1} + l_{m-1} - n_{m-1} \]
\item Die insgesamt produzierte Menge an Eiscreme muss gleich der gesamten Nachfrage sein:
\[ \sum p_m = \sum n_m \]
\item Die Änderung der Produktionsmenge $c_m$ ergibt sich aus der Differenz der Produktionsmenge des Monates und des vorherigen Monates: 
\[ c_m = p_{m} - p_{m-1} \] 
	\begin{itemize}
	\item Man muss hierbei beachten dass sowohl negative als auch positive Änderungen kosten! Da man innerhalb eines Linearen Problems jedoch nur mit linearen Funktionen, d.h. ohne $\text{abs()}$ arbeiten muss, müssen zwei Hilfsvariablen, $cd_m$ und $ci_m$ (decrease und increase) aufgestellt werden:
	\begin{align*}
	c_m &= ci_m - cd_m \\
	ci_m &\geq 0		\\
	cd_m &\geq 0
	\end{align*}
	
	\item Die absolute Gesamtänderung der Produktionsrate kann nun mit $ci_m + cd_m$ errechnet werden.
 	\end{itemize}
\end{itemize}

Zudem will man die Gesamtkosten minimieren, d.h. :
\[ \min: \sum (1 \cdot p_m + 50 \cdot (ci_m + cd_m) + 20 \cdot l_m)\]

Will man dies nun mithilfe eines Solvers lösen wollen, so würde eine mögliche Implementation der Bedingungen folgendermaßen aussehen:
\newline

min $\sum\{m~in~M\} ~ (1 \cdot p[m] + 50 \cdot (ci[m] + cd[m]) + 20 \cdot l[m])$ \\
s.t. $\{m~in~M\} ~ p[m] + l[m] \geq n[m]$ \\
s.t. $\{m~in~M\} ~ l[m] = p[m - 1] + l[m - 1] - n[m - 1]$ \\
s.t. $\{m~in~M\} ~ \sum p[m] - n[m] = 0$ 	\\
s.t. $\{m~in~M\} ~ c[m] = p[m] - p[m-1]$ 	\\
s.t. $\{m~in~M\} ~ cd[m] \geq 0$			\\
s.t. $\{m~in~M\} ~ ci[m] \geq 0$			\\
s.t. $\{m~in~M\} ~ c[m] = ci[m] - cd[m]$	
\newline

solve;