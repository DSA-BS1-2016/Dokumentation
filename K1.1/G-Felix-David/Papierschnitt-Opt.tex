Ein weiteres Problem mit welchem wir uns befassten war das Problem einer theoretischen Papierfirma. Besagte Firma produziert Standard-Papierrollen mit einer Breite von 3m, bietet jedoch für ihre Kunden verschiedene andere Schnittbreiten an, von denen jeweils eine bestimmte Menge bestellt wurde: \newline
{\centering
\begin{tabular}{|c|c|}
\hline Schnittbreite & Anzahl \\
\hline 135cm & 97-mal \\ 
\hline 108cm & 610-mal \\ 
\hline 93cm & 395-mal \\ 
\hline 42cm & 211-mal \\ 
\hline 
\end{tabular}}

\vspace{10pt}

Das Problem nun befasst sich damit, wie man möglichst wenige Papierrollen benötigt, um alle Bestellungen zu decken.

Nun müssen einige Variablen festgelegt werden:

\begin{itemize}
\item \emph{Menge S:} Die Menge $S$ beschreibt die Anzahl aller möglichen verschiedenen Schnittmuster, welche auf eine Rolle gebracht werden können. Dies sind insgesamt 12 diverse Kombinationen. Diese wären:

\begin{tabular}{|c|c|c|c|c|}
\hline Schnittmuster & 135\text{cm} & 108\text{cm} & 93\text{cm} & 42\text{cm} \\ 
\hline 1 & 2 & 0 & 0 & 0  \\ 
\hline 2 & 1 & 1 & 0 & 1 \\ 
\hline 3 & 1 & 0 & 1 & 1  \\ 
\hline 4 & 0 & 0 & 0 & 3 \\ 
\hline 5 & 0 & 2 & 0 & 2 \\ 
\hline 6 & 0 & 1 & 2 & 0  \\ 
\hline 7 & 0 & 1 & 1 & 2  \\ 
\hline 8 & 0 & 1 & 0 & 4  \\
\hline 9 & 0 & 0 & 3 & 0  \\ 
\hline 10 & 0 & 0 & 2 & 2  \\ 
\hline 11 & 0 & 0 & 1 & 4  \\ 
\hline 12 & 0 & 0 & 0 & 7  \\ 
\hline 
\end{tabular} 

\item Zu einem Element der Menge $S$, $s \in S$, liegen nun verschiedene Werte vor:
\begin{itemize}
\item Die Häufigkeit $h$ eines Schnittmusters $s$, d.h. $h_s$, gibt an, wie häufig das besagte Schnittmuster verwendet wird.
\item Die Anzahl $a$ eines Schnittmusters $s$ für eine gewisse Schnittbreite $b$, d.h. $a_{sb}$ gibt an, wie oft die Schnittbreite $b$ im Schnittmuster $s$ vorkommt.
\end{itemize}
\item \emph{Menge B:} Die Menge $B$ beschreibt die vier verschiedenen Schnittbreiten. Sie kann ausgedrückt werden als: 
\[ B = \{135\text{cm},108\text{cm},93\text{cm},42\text{cm}\} \]
\begin{itemize}
\item \emph{Nachfrage n:} Die Nachfrage $n \in N$ einer Schnittbreite $b \in B$, d.h. $n_b$, gibt an, wie viele Elemente der jeweiligen Schnittbreite $b$ bestellt wurden. 
\end{itemize}
\end{itemize}

Das gesamte Lineare Programm sieht nun folgendermaßen aus:
 
\begin{align*}
\min p & \\
 n_b  & \leq \sum\{s \in S\}a_{sb} h_s\\
h_s & \geq 0 \\
p & = \sum\{s \in S\}h_s 
\end{align*}

Im genaueren:

\begin{itemize}
\item Die Nachfrage einer Bestellung einer Schnittbreite $n_b$ muss erfüllt werden. Hierzu nimmt man die Summe aller Schnittmuster $s \in S$ multipliziert mit der Anzahl $a_sb$ der Schnittbreite $b$ welche sie enthalten multipliziert mit der Häufigkeit $h_s$ des Schnittmusters.
\[ \sum\{s \in S\}a_{sb} h_s \geq n_b \] 
\item Zudem muss noch festgelegt werden, dass ein Schnittmuster $s$ nicht weniger als 0 mal vorkommen kann:
\[ h_s \geq 0 \]
\item Die insgesamt verwendete Anzahl an Papierrollen ergibt sich nun aus der Summe der verwendeten Schnittmuster:
\[ p = \sum\{s \in S\}h_s \]
\end{itemize}
 
