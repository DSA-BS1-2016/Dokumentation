Das vierte Beispiel-Problem mit welchem wir uns befasst haben war das Problem einer theoretischen Papierfirma. Besagte Firma produziert Standard-Papierrollen mit einer Breite von 3m, bietet jedoch für ihre Kunden verschiedene andere Schnittbreiten an, von denen jeweils eine bestimmte Menge bestellt wurde:
\begin{itemize}
\item 97 Rollen mit 135cm Breite
\item 610  Rollen mit 108cm Breite
\item 395  Rollen mit 93cm Breite
\item 221  Rollen mit 42cm Breite
\end{itemize}

Das Problem nun befasst sich damit, auf welche Art und Weise man einen möglichst geringen Verlust an Papier durch die Schnitte hat.
Nun können schnell einige erste Bedingungen aufgestellt werden:

\begin{itemize}
\item Die Gesamtmenge einer Bestellung einer Schnittbreite $n_b$ muss erfüllt werden. Hierzu nimmt man die Summe aller Schnittmuster $s \in S$ multipliziert mit der Anzahl $a_sb$ der Schnittbreite $b$ welche sie enthalten multipliziert mit der Häufigkeit $h_s$ des Schnittmusters.

\[ \sum\{s \in S\}a_{sb} h_s \geq n_b \]
\item Zudem kann der Gesamtverlust $v$ definiert werden als der Verlust $w_s$ pro Schnittmuster $s \in S$ multipliziert mit der Häufigkeit dieses Schnittmusters.
\[ v = \sum\{s \in S\}w_s h_s  \] 
\item Als letztes muss noch festgelegt werden, dass ein Schnittmuster $s$ nicht weniger als 0 mal vorkommen kann:
\[ h_s \geq 0 \]
\end{itemize}

Nun kann man diese Formeln in den Simplex eintragen, wobei der Gesamtverlust $v$ möglichst minimiert werden soll: 
\[ \min v \] 