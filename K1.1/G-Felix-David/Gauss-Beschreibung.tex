Der Gauss-Algorithmus ist eine Möglichkeit, eine gegebene Matrix $M$ auf Zeilen-Stufen-Form zu bringen.
Dies bedeutet, dass innerhalb der Matrix $M$ die Anzahl der Nullen, welche bis zum ersten Nicht-Null-Element führen, größer der Anzahl der führenden Nullen der oberen Reihe sein muss.

Eine derart umgeformte Matrix M mag z.~B. so aussehen:
\[ M = \begin{pmatrix}
5 & 6 & 9 & 8\\ 
0 & 7 & 5 & 1\\ 
0 & 0 & 0 & 3
\end{pmatrix} \]

Hierbei heißen die jeweiligen Spalten, in denen das erste Nicht-Null-Element liegt Pivot-Elemente (Hier wären dies z.~B. die Spalten 1, 2 und 4).

Aus der Zeilen-Stufen-Form lässt sich zudem der Rang, und somit die Dimension, der Matrix $M$ an der Anzahl der Pivot-Spalten ablesen (Hier z.~B. 3)\footnote{Mehr zum Rang einer Matrix ist im nächsten Kapitel gegeben.}.
Auch sind die einzelnen Nicht-Null-Zeilenvektoren, sowie die Spalten-Vektoren der Pivot-Elemente, voneinander linear Unabhängig.

Es ist demnach schnell ersichtlich dass der Gauss-Algorithmus ein essentielles Element beim arbeiten mit Matrizen darstellt.